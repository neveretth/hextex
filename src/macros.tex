\newcommand{\addfig}[4]{
\begin{figure}[H]
\vspace{0mm}
\noindent\minipage{\linewidth}
\begin{center}
  \scalebox{#2}{
    \includegraphics[angle=#3, width=\textwidth]{#1}
  }\\
\end{center}
\endminipage
\vspace{0mm}
\caption{#4}
\vspace{-2mm}
\end{figure}
}

\renewcommand{\table}[4]{
\begin{figure}[H]
\vspace{0mm}
\noindent\minipage{\linewidth}
\begin{center}
  \scalebox{#1}{
    \begin{tabular}{#2}
      #3
    \end{tabular}
  }\\
\end{center}
\endminipage
\vspace{0mm}
\caption{#4}
\vspace{-2mm}
\end{figure}
}

\newcommand{\eq}[1]{
\vspace{-4mm}
\begin{equation}
\begin{split}
  #1
\end{split}
\end{equation}
}

\renewcommand{\l}[0]{
  \left(
}

\renewcommand{\r}[0]{
  \right)
}

% Probably don't use this, but it's here if you want something __super__ simple.
\newcommand{\circuit}[2]{
\begin{figure}[H]
\vspace{0mm}
\noindent\minipage{\linewidth}
\begin{center}
  \scalebox{1}{
    \begin{circuitikz}[american voltages] \draw
      #1
    ;
    \end{circuitikz}
  }\\
\end{center}
\endminipage
\vspace{0mm}
\caption{#2}
\vspace{-2mm}
\end{figure}
}

% I cannot fucking _believe_ this is the syntax for hrule.
\renewcommand{\rule}{
  \vspace{4mm}
  \hrule height 0.25pt
  \vspace{4mm}
}
